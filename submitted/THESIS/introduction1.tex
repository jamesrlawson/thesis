%
%%%%%%%%%%%%%%%%%%%%%%%%%%%%%%%%%%%%%%%%%%%%%%%%%%%%%%%%%%%%%%%%%%%%%%
% Tina Dissertation
% December 2013, modified to Template June 2015
%%%%%%%%%%%%%%%%%%%%%%%%%%%%%%%%%%%%%%%%%%%%%%%%%%%%%%%%%%%%%%%%%%%%%%
% Documentclass Memoir 
% check memman.pdf for help and information
%%%%%%%%%%%%%%%%%%%%%%%%%%%%%%%%%%%%%%%%%%%%%%%%%%%%%%%%%%%%%%%%%%%%%%
\documentclass[openright,12pt,a4paper]{memoir} 
\usepackage{graphicx}
%\usepackage[utf8]{inputenc} % set input encoding to utf8
\usepackage{array} % for tables 
\usepackage{multirow} % for tables 
\usepackage{multicol} % for tables
\usepackage{tabularx} % for tables
\usepackage{booktabs}
\usepackage{cite}
\usepackage{tabularx}
\usepackage[round]{natbib}
\usepackage{threeparttable}
\DisemulatePackage{setspace}
\usepackage{setspace}
\usepackage{longtable}
\usepackage{tabu}
\usepackage{pdflscape}
\usepackage{caption}
%\usepackage{lmodern}
\usepackage{url} \usepackage[normalem]{ulem}
\useunder{\uline}{\ul}{}



% defines new column type
\newcolumntype{Z}{$>${\raggedright\arraybackslash}X}

% add a little vertical padding to cramped tables
\setlength{\extrarowheight}{2pt}


%%%%%%%%%%%%%%%%%%%%%%%%%%%%%%%%%%%%%%%%%%%%%%%%%%%%%%%%%%%%%%%%%%
%%% Examples of Memoir customization
%%% enable, disable or adjust these as desired

%%% PAGE DIMENSIONS
% a4paper is by default 210mm wide and 279 mm wide

% default document in memoir is twoside (recto-verso) and openright (new chapter begins on recto page)

% size of the text area
  \settrims{0pt}{0pt}
  \settypeblocksize{230mm}{147mm}{*}
  \setlength{\spinemargin}{27mm}
  \setlength{\foremargin}{36mm}
%\setulmargins{35mm}{45mm}{*}
%\setlength{\marginparwidth}{0mm}
%\setlength{\marginparsep}{0mm}
%\setlength{\textwidth}{140mm}
%\settrimmedsize{0.9\stockheight}{0.9\stockwidth}{*}
%\setlength{\trimtop}{0pt}
%\setlength{\trimedge}{0pt}
%\addtolength{\trimedge}{-\paperwidth}
%\settypeblocksize{*}{\lxvchars}{1.618} % we want to the text block to have golden proportionals
  \setulmargins{*}{*}{1.618} % 50pt upper margins
%\setlrmargins{*}{*}{1.3}
%  \setlrmargins{*}{*}{1} % golden ratio again for left/right margins
  \setheaderspaces{*}{*}{1.618}
  \checkandfixthelayout % to make sure that the layout parameters make sense

%\addtolength{\textwidth}{0cm}
%\addtolength{\textheight}{1.5cm}
%\addtolength{\textwidth}{-2cm}
%\addtolength{\textheight}{+0.5cm}

%%% \maketitle CUSTOMISATION
% For more than trivial changes, you may as well do it yourself in a titlepage environment
%\pretitle{\begin{center}\sffamily\Huge\MakeUppercase}
%\posttitle{\par\end{center}\vskip 0.5em}

%%% ToC (table of contents) APPEARANCE
  \maxtocdepth{subsection} % include subsections
%\renewcommand{\cftchapterpagefont}{}
%\renewcommand{\cftchapterfont}{}     % no bold!

%%% HEADERS & FOOTERS
  \pagestyle{headings} % try also: empty , plain , headings , ruled , Ruled , companion

%%% CHAPTERS
  \chapterstyle{southall} % try also: default , section , hangnum , companion , article, demo

  \renewcommand{\chaptitlefont}{\LARGE\sffamily\raggedright} % set sans serif chapter title font
  \renewcommand{\chapnumfont}{\LARGE\sffamily\raggedright} % set sans serif chapter number font

%%% TABLES
  \newcolumntype{C}[1]{$>${\centering}m{#1}} % defines the default layout of the tables (C$=$centerling, L$=$left)
  \newcolumntype{L}[1]{$>${\centering}m{#1}}

%%% SECTIONS
%\hangsecnum % hang the section numbers into the margin to match \chapterstyle{hangnum}
  \maxsecnumdepth{section} % number subsections

  \setsecheadstyle{\Large\sffamily\raggedright} % set sans serif section font
  \setsubsecheadstyle{\large\sffamily\raggedright} % set sans serif subsection font

%%% Abstract
  \setlength{\absleftindent}{0mm}
  \setlength{\absrightindent}{0mm}

  \renewcommand{\absnamepos}{center}
  \setlength{\abstitleskip}{+0cm}

%%% Captions

%\DeclareCaptionFont{tiny}{\tiny}
%\captionsetup{font$=$tiny, labelfont$=$tiny}
%\usepackage[font$=${tiny}, labelfont$=${tiny}]{caption}
%\usepackage[font$=$sf, labelfont$=${sf,bf}, margin$=$1cm]{caption}
%\captionsetup{font$=$scriptsize,labelfont$=$scriptsize}

%\usepackage[textfont$=${tiny}, labelfont$=${tiny}]{caption}

 % \captionnamefont{\tiny}
 %\captiontitlefont{\tiny}

%% END Memoir customization


%%%%%%%%%%%%%%%%%%%%%%%%%%%%%%%%%%%%%%%%%%%%%%%%%%%%%%%%%%%%%%%%%%%%%%%%%%%%%%%%%%%%%%%%%%%%%%%%%%%%%%%%%%%%%%%%%%%%%%%%%%%%%%%%%%%%%%%%%%%%%
%%% BEGIN DOCUMENT

\begin{document}
\doublespacing


\chapter[Introduction]{Introduction}
\newpage

\noindent{Riparian ecosystems are biophysically complex and highly diverse taxonomically, structurally and functionally \citep{Naiman1993, Poff2002, Nilsson2002}. They provide a disproportionately high quantity of ecosystem goods and services relative to the fraction of the landscape that they occupy \citep{Capon2013} and play a critical role in maintaining regional biodiversity \citep{Naiman1993}. Riparian landscapes have been heavily modified by humans; in many parts of the world this modification has taken place rapidly and has resulted in significant habitat degradation and biodiversity loss \citep{Arthington2010}. In particular, catchment-scale clearing of vegetation, impoundment and flow regulation have altered the hydrology of river systems globally \citep{Nilsson2000}. As demand for water increases with growing human populations, river systems are likely to become increasingly modified. Changing climatic conditions over the next century are also expected to cause shifts in hydrological patterns. Predictions are regionally specific but typically include changes to total discharge, flow seasonality and flow variability \citep{stocker2013climate}. In regions with projected increases in climatic variability, changes to the prevalence, intensity and timing of extreme flooding or drought events can be expected \citep{Hennessy2008}. This combination of flow regulation and alterations to baseline discharges may well produce dramatically altered future flow regimes, with significant consequences for the diversity and functional composition of riparian vegetation communities \citep{Poff2010}.}
Due to their central role in maintaining diversity and ecosystem functioning at the landscape scale \citep{Naiman1993}, riparian ecosystems are the target of substantial management effort \citep{Goodwid1997}. Vegetation assemblages receive particular attention in riparian management as they set the coarse physical structure of biotic communities and play an important role in generating and maintaining the characteristic geomorphology of river systems \citep{Richardson2007, Corenblit2007}. Conservation planning in fluvial landscapes therefore requires context-specific understanding of environmental controls on plant community assembly. 

\section{The importance of flow regime to riparian plant communities}
Flow regime is thought to be a dominant abiotic control on the composition and structure of riparian plant communities \citep{Poff1997}. Stream flows affect plant communities directly by causing flooding disturbance and driving variation in nutrient and moisture availability \citep{Naiman1997}, as well as by interaction with sediment and geomorphic processes \citep{Corenblit2007}. The inherently heterogeneous nature of fluvial interaction with vegetated landforms results in structurally complex, patchy landscapes containing strong energy and resource gradients \citep{Naiman2005}. Spatial and temporal heterogeneity in the magnitude, frequency, duration, timing, rates of change and predictability of flow discharge \citep{Poff1997, Kennard2010} translates to heterogeneous influence of stream flows on riparian patches \citep{Poff1997, Naiman2008}. Niche-oriented models of riparian ecology  hold that it is this environmental heterogeneity which supports the high degree of biodiversity observed in riparian ecosystems \citep{Palmer1997, Bornette2008}.

Given the profound influence of fluvial hydrology on riparian vegetation communities, determining specific flow-ecology relationships has long been a goal in riparian research \citep{Auble1994, Lytle2004}. To date, this research has been largely driven by work on impacts of dams on vegetation \citep{Goodwid1997, Nilsson2002}. The resulting insight into the comparative hydroecology of \textit{Populus} and \textit{Salix spp.} and invasive \textit{Tamarix spp.} in North American river systems \citep{Mahoney1998, Shafroth2002} has led to significant advances in design of environmental flows to support indigenous vegetation assemblages \citep{Shafroth2010}. This approach is effective in western North America, where well-understood systems are dominated by a limited set of species. However, approaches centred on deterministic species-specific flow response models are less practical in more diverse or less understood systems. 

\section{The intersection between hydroecology and functional ecology}
Maintenance of biodiversity and ecosystem functioning may be a more appropriate conservation target than supporting the persistence of specific assemblages \citep{Aerts2011, Cadotte2011, Montoya2012} given the predictions of future climate and other environmental change. Where sites harbour dissimilar species assemblages, comparison becomes challenging. Taxonomic descriptors of communities such as species richness or species diversity are widely used to compare communities across landscapes, but are unable to provide information about how elements of a community influence ecosystem functioning, provision of ecosystem services, or contribute to system resilience \citep{Tilman1997, Diaz1998, Diaz2007}. Describing communities in terms of functional traits - any morphological, physiological or phenological feature measurable at the individual level \citep{Violle2007} - dissolves species distinctions and emphasises ecological strategies: what species do within their community and how they do it \citep{Diaz1998}. This allows dissimilar communities to be compared in terms of how their component species both respond to and have an effect on their environment \citep{Lavorel2002, Suding2008}. A functional trait oriented approach applied to flow-vegetation interactions facilitates the search for regional generalities in hydrological controls on ecosystem processes and patterns of diversity.

Species are different, but not equally so, and the nature and extent of species differences characterises ecological communities. Data about appropriately selected quantitative functional traits (such as specific leaf area, wood density, seed mass etc.) can form the basis for mechanistic assessments of diversity which describe the range and distribution of ecological strategies in a community and their associated environmental effects \citep{Schleuter2010}. These indices of functional trait diversity are increasingly being employed in ecosystem assessment and management as a complement to traditional taxonomic metrics of diversity \citep{Tilman1997, Lavorel2013}.

Hydroecologically derived plant functional groups have been described for some time \citep{Stromberg2010, Casanova2011}, but advances in quantitative plant ecology based on functional traits are only beginning to be applied to riparian systems. Notable early contributions to the quantitative riparian functional hydroecology literature include discussion of variation in functional traits according to species origin (i.e. native or exotic), geomorphology and fluvial disturbance \citep{Kyle2009, Kyle2009a}, and evidence for reduced functional trait diversity in riparian wetlands in response to flow impoundment \citep{Catford2011}. Merritt et al. \citep{Merritt2010} outlined a framework for defining riparian vegetation flow response guilds according to functional traits, and functional traits have been discussed as a means by which to predict riparian community responses to climate change \citep{Catford2012a, Kominoski2013}. Momentum is now building for insights from plant functional ecology to be applied to riparian conservation planning and management.

\section{Anthropogenic impacts on riparian plant communities}
Rapid development of catchments has changed fundamental processes which create and maintain biodiversity in riparian ecosystems \citep{Nilsson2002}, and as such, riparian management often takes place within this context of catchment modification. Environmental homogenisation of riparian landscapes by flow modification, land-use change and invasion by exotic plants has profound implications for riparian biodiversity \citep{Brierley1999, Richardson2007, Poff2010}. 

Given the dominance of flow regime in shaping and driving riparian ecosystems, flow modification is likely to have the greatest impact, although anthropogenic stressors are typically not independent from each other. Dams, weirs and diversions affect river systems in populated regions worldwide \citep{Nilsson2002}, resulting in diminished discharge, reduced flow variability, dampening of flood peaks and changes to seasonality of flows \citep{Graf2006, Singer2007}. Depending on the magnitude of change, biogeomorphic simplification and weedy invasion may occur downstream of dams \citep{Graf2006, Naiman2008, Catford2011}. The impact of these changes on riparian plant communities is likely to be compounded by the deleterious effects of land transformation, primarily habitat fragmentation and loss of catchment-scale alpha and beta diversity \citep{Vitousek1997, Gerstner2014}. Exotic invasion is closely associated with human activity \citep{Vitousek1996} and itself represents a significant threat to riparian plant communities \citep{Richardson2007}.

Rising atmospheric carbon dioxide represents a further unexplored variable with the potential to alter future riparian plant communities. Atmospheric CO\textsubscript{2} has risen substantially over the past century and a doubling of pre-industrial levels by 2100 is projected \citep{IPCC2014}. A substantial body of research describes dramatic effects of elevated CO\textsubscript{2} (eCO\textsubscript{2}) on plant growth, physiology and community ecology \citep{Curtis1996, Poorter2003a, Reich2014}. Typical responses to eCO\textsubscript{2} include stimulation of photosynthetic carbon assimilation \citep{Curtis1996}, reduced stomatal conductance \citep{Ainsworth2007}, greater water use efficiency \citep{Holtum2010, VanderSleen2014}, greater biomass accumulation \citep{Wang2012}, altered biomass allocation \citep{Nie2013}, and changes to functional traits indicative of positions along economic spectra (i.e. slow vs fast growth strategy) \citep{Poorter2003a, Bader2010}. The effects of eCO\textsubscript{2} vary between species and are often contingent on other environmental variables such as availability of water and macronutrients \citep{Korner2006, Manea2014, Reich2014}.
 
Plants growing adjacent to stream channels enjoy the best access to water in the landscape, but the privilege is not free. Along with exposure to flooding disturbance, most riparian plants must at some point endure waterlogging or inundation \citep{Tabacchi1998, Colmer2009}. Inundation represents a significant stress to riparian plants, and root zone anoxia is well established as the mechanism driving plant physiological and functional responses to waterlogging, impairing root metabolism and uptake of water and nutrients \citep{Drew1997, Piedade2010, Voesenek2015}, altering root traits \citep{Steffens2013} and disrupting mutualisms with soil biota \citep{Dawson1989, Shimono2012}.

Elevated atmospheric levels of CO\textsubscript{2} and inundation appear likely to have opposing effects on plant growth, but the possibility that eCO\textsubscript{2} may mitigate growth reduction under waterlogging warrants investigation of the interactive effects of these two important environmental variables. The limited literature describing the interactive effects of eCO\textsubscript{2} and waterlogging or inundation on plant growth presents an inconsistent picture, with effects varying widely between species \citep{Megonigal2005, Shimono2012, Arenque2014}. Generation of harmful reactive oxygen species has been shown to accompany reaeration after waterlogging \citep{Drew1997}, and as such recovery from flooding represents a different stress to tolerance of the event itself. To date, no research has described the effects of eCO\textsubscript{2} on recovery from waterlogging.

\section{Research questions}
In this thesis, I set out to identify the fundamental relationships between riparian plant communities and the various environmental controls and stresses which define them. To this end, I used concepts and tools from modern functional ecology to determine the mechanisms by which communities organise themselves along gradients of these environmental conditions.

In Chapter 2, I asked how wood density, a key plant functional trait integrating the trade-off between rapid growth and tolerance of physical disturbance and drought, varies along gradients of fluvial disturbance intensity and variability in water availability across 15 sites in natural landscapes of south-eastern Australia. Chapter 3 extended this research question, using the same set of sites to investigate relationships between functional diversity and environmental heterogeneity. In Chapter 4, data from sites spanning gradients of flow modification and land-use intensity in south-east Queensland, Australia, were analysed to determine the relative importance of natural and anthropogenic controls on taxonomic and functional diversity. This chapter is again guided by hypotheses about relationships between environmental variability and diversity. In Chapter 5, I described a manipulative glasshouse experiment designed to assess how elevated atmospheric concentrations of CO\textsubscript{2} might affect future responses of riparian trees to waterlogging. Finally, in Chapter 6 (Discussion), I summarise my key findings, assess the contribution of my thesis to the broader literature, and suggest avenues for future research.

%%%%% REFERENCES % this is in a new chapter due to the memoir format
\renewcommand\bibname{{References}} 
\begin{small}
\bibliographystyle{apalike}
\bibliography{library}
\end{small}

\end{document}


\chapter[Appendix 2a]{Appendix 2a (supplementary to Chapter 3)}

\begin{landscape}
\begin{table}[ht]
\tiny
\centering
\caption[Location and characteristics of field sites.]{\small{Location and characteristics of field sites.}} \\
\label{Ch3sup1_T1}
{\tabulinesep=1.2mm
%\begin{tabu} to \linewidth {m{3.2cm}m{5cm}X}
\begin{tabu} to \linewidth {lm{8cm}XXXX}
\hline
\textit{Site} & \textit{Gauge Name} & \textit{Longitude} & \textit{Latitude} & \textit{Catchment area (km\textsuperscript{2})} & \textit{Elevation (m asl)} \\
\hline
1 & Mammy Johnsons River at Pikes Crossing & 151.979 & -32.244 & 158 & 104 \\
2 & Wallagaraugh River at Princes Highway & 149.714 & -37.371 & 477 & 35 \\
3 & Genoa River at Bondi & 149.321 & -37.174 & 234 & 417 \\
4 & Wadbilliga River at Wadbilliga & 149.694 & -36.259 & 126 & 201 \\
5 & Tuross River D/S Wadbilliga Junction & 149.761 & -36.197 & 918 & 79 \\
6 & Tuross River at Belowra & 149.709 & -36.201 & 564 & 105 \\
7 & Jacobs River at Jacobs Ladder & 148.427 & -36.727 & 184 & 343 \\
8 & Nariel Creek at Upper Nariel & 147.826 & -36.444 & 261 & 711 \\
9 & Gibbo River at Gibbo Park & 147.709 & -36.756 & 390 & 515 \\
10 & Snowy Creek at Below Granite Flat & 147.413 & -36.569 & 416 & 331 \\
11 & Mann River at Mitchell & 152.105 & -29.695 & 890 & 401 \\
12 & Cataract Creek at Sandy Hill & 152.217 & -28.934 & 237 & 595 \\
13 & Sportsmans Creek at Gurranang Siding & 152.981 & -29.467 & 205 & 13 \\
14 & Goodradigbee River at Brindabella & 148.731 & -35.421 & 432 & 510 \\
15 & Jilliby Creek at U/S Wyong River & 151.389 & -33.246 & 93 & 39 \\
\hline
\end{tabu}}
\end{table}
\end{landscape}
\clearpage


\begin{table}[ht]
\tiny
\centering
\caption[Importance of principal components (hydrology PCA).]{\small{Importance of components, from Principal Components Analysis of the set of 23 hydrological metrics used as explanatory variables in this study.}}\\
\label{Ch3sup1_T2}
{\tabulinesep=1.2mm
%\begin{tabu} to \linewidth {m{3.2cm}m{5cm}X}
\begin{tabu} to \linewidth {lm{3cm}XXXX}
\hline
& \textit{PC1} & \textit{PC2} & \textit{PC3} & \textit{PC4} & \textit{PC5}   \\
\hline
Standard deviation & 3.848 & 1.824 & 1.377 & 0.935 & 0.788 \\
Proportion of variance & 0.644 & 0.145 & 0.082 & 0.038 & 0.027 \\
Cumulative proportion & 0.644 & 0.788 & 0.871 & 0.909 & 0.936 \\
\hline
\end{tabu}}
\end{table}

\begin{table}[ht]
\tiny
\centering
\caption[Loadings across principal components (hydrology PCA).]{\small{Loadings across principal components for the set of 23 hydrological metrics used in this study.}}\\
\label{Ch3sup1_T3}
{\tabulinesep=1.2mm
%\begin{tabu} to \linewidth {m{3.2cm}m{5cm}X}
\begin{tabu} to \linewidth {m{4cm}XXXXX}
\hline
\textit{metric} & \textit{PC1} & \textit{PC2} & \textit{PC3} & \textit{PC4} & \textit{PC5} \\
\hline
HSPeak & -0.24 & 0.09 & 0.23 & 0.13 & 0.02 \\
MDFAnnHSNum & -0.08 & -0.46 & -0.08 & -0.39 & 0.03 \\
CVAnnHSNum & 0.00 & 0.40 & -0.34 & 0.43 & 0.06 \\
CVAnnHSPeak & -0.19 & 0.06 & -0.42 & -0.22 & 0.03 \\
MRateRise & -0.22 & -0.18 & 0.20 & 0.18 & 0.12 \\
MRateFall & -0.21 & -0.25 & 0.10 & 0.22 & -0.02 \\
CVAnnMRateRise & -0.22 & 0.26 & -0.02 & -0.15 & 0.02 \\
CVAnnMRateFall & -0.24 & 0.14 & 0.08 & -0.05 & 0.22 \\
AS20YrARI & -0.25 & 0.02 & 0.01 & 0.05 & -0.02 \\
C\_MDFM & 0.24 & -0.10 & -0.14 & 0.12 & 0.24 \\
M\_MDFM & 0.25 & 0.01 & 0.02 & 0.09 & -0.30 \\
C\_MinM & 0.24 & -0.07 & -0.15 & 0.19 & 0.14 \\
M\_MinM & 0.22 & 0.11 & 0.10 & 0.10 & -0.53 \\
C\_MaxM & 0.02 & -0.43 & 0.33 & 0.27 & 0.10 \\
M\_MaxM & 0.25 & -0.02 & 0.00 & 0.07 & -0.01 \\
MDFMDFSpring & 0.24 & 0.08 & 0.11 & -0.19 & 0.19 \\
MDFMDFSummer & -0.18 & -0.19 & -0.44 & 0.19 & 0.07 \\
MDFMDFAutumn & -0.23 & -0.13 & -0.04 & 0.17 & -0.42 \\
MDFMDFWinter & 0.14 & 0.30 & 0.40 & -0.13 & 0.23 \\
CVMDFSpring & -0.22 & 0.10 & 0.20 & 0.35 & 0.14 \\
CVMDFSummer & -0.22 & 0.11 & 0.14 & -0.20 & -0.42 \\
CVMDFAutumn & -0.24 & 0.02 & 0.01 & -0.24 & 0.09 \\
CVMDFWinter & -0.21 & 0.22 & 0.03 & 0.08 & 0.06 \\
\hline
\end{tabu}}
\end{table}

\begin{table}[ht]
\tiny
\centering
\caption[Summary statistics for hydrological variables.]{\small{Summary statistics for hydrological variables. From left: minimum, maximum, mean and standard deviation.}}\\
\label{Ch3sup1_T4}
{\tabulinesep=1.2mm
%\begin{tabu} to \linewidth {m{3.2cm}m{5cm}X}
\begin{tabu} to \linewidth {m{4cm}XXXX}
\hline
\textit{metric} & \textit{min} & \textit{max} & \textit{mean} & \textit{sd} \\
\hline
HSPeak & 5.38 & 29.81 & 16.67 & 8.34 \\
MDFAnnHSNum & 2.8 & 5.93 & 4.1 & 0.96 \\
CVAnnHSNum & 0.48 & 0.84 & 0.74 & 0.11 \\
CVAnnHSPeak & 0.24 & 1.47 & 0.69 & 0.34 \\
MRateRise & 0.2 & 1.99 & 0.91 & 0.57 \\
MRateFall & 0.07 & 0.8 & 0.34 & 0.23 \\
CVAnnMRateRise & 0.43 & 1.18 & 0.85 & 0.25 \\
CVAnnMRateFall & 0.41 & 1.46 & 0.9 & 0.34 \\
AS20YrARI & 17.94 & 209.99 & 126.13 & 81.19 \\
C\_MDFM & 0.05 & 0.31 & 0.14 & 0.09 \\
M\_MDFM & 0.06 & 0.2 & 0.12 & 0.05 \\
C\_MinM & 0.01 & 0.27 & 0.12 & 0.08 \\
M\_MinM & 0.07 & 0.16 & 0.11 & 0.03 \\
C\_MaxM & 0.19 & 0.44 & 0.28 & 0.09 \\
M\_MaxM & 0.04 & 0.18 & 0.09 & 0.06 \\
MDFMDFSpring & 0.19 & 1.81 & 1.02 & 0.55 \\
MDFMDFSummer & 0.42 & 1.49 & 0.88 & 0.33 \\
MDFMDFAutumn & 0.28 & 1.82 & 1 & 0.52 \\
MDFMDFWinter & 0.64 & 1.44 & 1.08 & 0.25 \\
CVMDFSpring & 0.36 & 2.1 & 1.12 & 0.54 \\
CVMDFSummer & 0.6 & 1.66 & 1.15 & 0.39 \\
CVMDFAutumn & 0.48 & 1.49 & 1.07 & 0.35 \\
CVMDFWinter & 0.46 & 1.99 & 1.05 & 0.46 \\
\hline
\end{tabu}}
\end{table}
\clearpage

\textbf{Bioclimatic variables assessed for relationships with FDis:}

\small{Annual Mean Temperature

Mean Diurnal Range (Mean of monthly (max temp - min temp))

Isothermality (BIO2/BIO7) (* 100)

Temperature Seasonality (standard deviation * 100)

Max Temperature of Warmest Month

Min Temperature of Coldest Month

Temperature Annual Range (BIO5-BIO6)

Mean Temperature of Wettest Quarter

Mean Temperature of Driest Quarter

Mean Temperature of Warmest Quarter

Mean Temperature of Coldest Quarter

Annual Precipitation

Precipitation of Wettest Month

Precipitation of Driest Month

Precipitation Seasonality (Coefficient of Variation)

Precipitation of Wettest Quarter

Precipitation of Driest Quarter

Precipitation of Warmest Quarter

Precipitation of Coldest Quarter}

\newpage

\textbf{Edaphic variables assessed for relationships with FDis:}

\small{Available Water Capacity

Bulk Density – Whole Earth (g cm\textsuperscript{-3})

Clay (\%)

Depth of Regolith (m)

Depth of Soil (m)

Effective Cation Exchange Capacity (meq / 100 g)

Total Nitrogen (\%)

pH – CaCl2 

Total Phosphorous (\%)

Silt (\%)

Sand (\%)

Organic Carbon (\%)}

\newline
Further information on soil variables can be found at: \newline
\url{http://www.clw.csiro.au/aclep/soilandlandscapegrid} (accessed 09/06/2015)

\chapter[Appendix 2b]{Appendix 2b (supplementary to Chapter 3)}

\begin{landscape}
\begin{table}[ht]
\tiny
\centering
\caption[Data density information for trait dataset.]{\small{Data density information for trait dataset. Coverage describes the total proportional coverage at a site for which species were included in the analysis. Density values for each trait describe the proportional coverage at a site for which data for that trait were included in the analysis. N.B. leaf narrowness and wood density were not available for grasses or ferns; seed mass and flowering period were also not available for ferns.}} \\
\label{Ch3sup2_T1} \\
{\tabulinesep=1.2mm
\begin{tabu} to \linewidth {XXXXXXXX}
\hline
\textit{site} & \textit{coverage} & \textit{wood density} & \textit{max. height} & \textit{seed mass} & \textit{SLA} & \textit{flowering period} & \textit{leaf narrowness} \\
\hline
1 & 0.98 & 0.615 & 1 & 0.846 & 1 & 0.923 & 0.692 \\
2 & 0.959 & 0.333 & 1 & 0.667 & 1 & 0.667 & 0.333 \\
3 & 0.949 & 0.455 & 1 & 0.727 & 1 & 0.727 & 0.545 \\
4 & 0.903 & 0.4 & 1 & 0.867 & 1 & 0.867 & 0.6 \\
5 & 0.968 & 0.455 & 1 & 1 & 1 & 1 & 0.545 \\
6 & 0.964 & 0.7 & 1 & 1 & 1 & 1 & 0.7 \\
7 & 1 & 0.5 & 1 & 1 & 0.9 & 1 & 0.7 \\
8 & 1 & 0.529 & 1 & 0.882 & 1 & 0.882 & 0.765 \\
9 & 0.988 & 0.474 & 1 & 0.842 & 1 & 0.842 & 0.737 \\
10 & 0.976 & 0.583 & 1 & 0.917 & 1 & 0.917 & 0.667 \\
11 & 0.96 & 0.188 & 1 & 1 & 0.938 & 1 & 0.688 \\
12 & 0.992 & 0.381 & 1 & 0.952 & 0.952 & 0.952 & 0.714 \\
13 & 0.935 & 0.55 & 0.95 & 0.9 & 1 & 0.9 & 0.7 \\
14 & 1 & 0.636 & 1 & 1 & 1 & 1 & 1 \\
15 & 0.963 & 0.455 & 1 & 0.909 & 0.909 & 0.909 & 0.727 \\
\hline
\end{tabu}}
\end{table}
\end{landscape}
\clearpage

\begin{table}[ht]
\tiny
\centering
\caption[Summary statistics for trait dataset.]{\small{Summary statistics for trait dataset. From left: minimum, maximum, mean and standard deviation.}} \\
\label{Ch3sup2_T2} \\
{\tabulinesep=1.2mm
\begin{tabu} to \linewidth {p{5cm}XXXX}
\hline
\textit{trait} & \textit{min} & \textit{max} & \textit{mean} & \textit{sd} \\
\hline
Max. height (m) & 0.2 & 50 & 10.47 & 13.18 \\
Seed mass (mg) & 0.04 & 323.99 & 16.55 & 45.06 \\
SLA (m\textsuperscript{2} / kg) & 1.41 & 63.27 & 17.93 & 14 \\
Flowering period length \newline(proportion of year) & 0.17 & 1 & 0.45 & 0.24 \\
Leaf narrowness (unitless ratio) & 0.59 & 233.33 & 9.86 & 32.53 \\
Wood density (g / cm\textsuperscript{3}) & 0.33 & 0.95 & 0.61 & 0.13 \\
\hline
\end{tabu}}
\end{table}


\begin{table}[ht]
\tiny
\centering
\caption[Importance of principal components (trait dataset, all traits).]{\small{Importance of principal components PC1 – PC5 from Principal Components Analysis of trait dataset, using species with data available for all traits (55 species).}} \\
\label{Ch3sup2_T3} \\
{\tabulinesep=1.2mm
\begin{tabu} to \linewidth {p{4cm}XXXXXX}
\hline
&                      \textit{PC1}  & \textit{PC2}    & \textit{PC3}    & \textit{PC4}    & \textit{PC5}    & \textit{PC6}            \\ 
\hline
Standard deviation     & 1.3938 & 1.0962 & 1.0827 & 0.9247 & 0.7438 & 0.52457 \\
Proportion of variance & 0.3238 & 0.2003 & 0.1954 & 0.1425 & 0.0922 & 0.04586 \\
Cumulative  proportion & 0.3238 & 0.5240 & 0.7194 & 0.8619 & 0.9541 & 1      \\
\hline
\end{tabu}}
\end{table}

\begin{table}[ht]
\tiny
\centering
\caption[Importance of principal components (trait dataset, only traits will full data coverage).]{\small{Importance of principal components PC1 – PC5 from Principal Components Analysis of trait dataset, using species with data available for SLA, maximum height, seed mass and flowering period length (97 species).}} \\
\label{Ch3sup2_T4} \\
{\tabulinesep=1.2mm
\begin{tabu} to \linewidth {p{4cm}XXXX}
\hline
& \textit{PC1}  & \textit{PC2}    & \textit{PC3}    & \textit{PC4}   \\
\hline
Standard deviation & 1.4160 & 1.0016 & 0.8326 & 0.54649 \\
Proportion of variance & 0.5012 & 0.2508 & 0.1733 & 0.07466 \\
Cumulative  proportion & 0.5012 & 0.7520 & 0.9253 & 1  \\
\hline
\end{tabu}}
\end{table}

\clearpage

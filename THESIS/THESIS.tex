
/documentclass[12pt,a4paper]{memoir}


%%%%%%%%%%%%%%%%%%%%%%%%%%%%%%%%%%%%%%%%%%%%%%%%%%%%%%%%%%%%%%%%%%%%%%
% Preamble and packages
\usepackage[utf8]{inputenc} % set input encoding to utf8
\usepackage{upgreek} % for Greek letters
\usepackage{textcomp}
\usepackage{array} % for tables 
\usepackage{multirow} % for tables 
\usepackage{multicol} % for tables
\usepackage{abstract} % to add an abstract before the main document begins
\usepackage{graphicx} % to add figures and graphs
\usepackage{float} % for placement of figures
\usepackage{anyfontsize} % font size
\usepackage{tabularx} % for tables
%\usepackage{pdflandscape}
\usepackage{lscape} % for tables in landscape format
\usepackage{pdfpages} % to include pdf documents (for example CV in the end)
\usepackage{longtable} % for tables that go more than one page
\usepackage{natbib} % bibliography style
\usepackage{url} % to add url and hyperlinks
\usepackage[sectionbib]{chapterbib} % so that references can be added to each individual chapter
\graphicspath{{\LateX_Figures}} % defines the path to the latex figures. This is needed if the figures are not in the same folder as the .tex file. 

%%%%%%%%%%%%%%%%%%%%%%%%%%%%%%%%%%%%%%%%%%%%%%%%%%%%%%%%%%%%%%%%%%
%%% Examples of Memoir customization
%%% enable, disable or adjust these as desired

%%% PAGE DIMENSIONS
% a4paper is by default 210mm wide and 279 mm wide

% default document in memoir is twoside (recto-verso) and openright (new chapter begins on recto page)

% size of the text area
\settrims{0pt}{0pt}
\settypeblocksize{230mm}{147mm}{*}
\setlength{\spinemargin}{27mm}
\setlength{\foremargin}{36mm}
%\setulmargins{35mm}{45mm}{*}
%\setlength{\marginparwidth}{0mm}
%\setlength{\marginparsep}{0mm}
%\setlength{\textwidth}{140mm}
%\settrimmedsize{0.9\stockheight}{0.9\stockwidth}{*}
%\setlength{\trimtop}{0pt}
%\setlength{\trimedge}{0pt}
%\addtolength{\trimedge}{-\paperwidth}
%\settypeblocksize{*}{\lxvchars}{1.618} % we want to the text block to have golden proportionals
\setulmargins{*}{*}{1.618} % 50pt upper margins
%\setlrmargins{*}{*}{1.3}
%\setlrmargins{*}{*}{1} % golden ratio again for left/right margins
\setheaderspaces{*}{*}{1.618}
\checkandfixthelayout % to make sure that the layout parameters make sense

%\addtolength{\textwidth}{0cm}
%\addtolength{\textheight}{1.5cm}
%\addtolength{\textwidth}{-2cm}
%\addtolength{\textheight}{+0.5cm}

%%% \maketitle CUSTOMISATION
% For more than trivial changes, you may as well do it yourself in a titlepage environment
%\pretitle{\begin{center}\sffamily\Huge\MakeUppercase}
%\posttitle{\par\end{center}\vskip 0.5em}

%%% ToC (table of contents) APPEARANCE
\maxtocdepth{subsection} % include subsections
%\renewcommand{\cftchapterpagefont}{}
%\renewcommand{\cftchapterfont}{}     % no bold!

%%% HEADERS & FOOTERS
\pagestyle{Ruled} % try also: empty , plain , headings , ruled , Ruled , companion

%%% CHAPTERS
\chapterstyle{southall} % try also: default , section , hangnum , companion , article, demo

\renewcommand{\chaptitlefont}{\LARGE\sffamily\raggedright} % set sans serif chapter title font
\renewcommand{\chapnumfont}{\LARGE\sffamily\raggedright} % set sans serif chapter number font

%%% TABLES
\newcolumntype{C}[1]{>{\centering}m{#1}} % defines the default layout of the tables (C=centerling, L=left)
\newcolumntype{L}[1]{>{\centering}m{#1}}

%%% SECTIONS
%\hangsecnum % hang the section numbers into the margin to match \chapterstyle{hangnum}
\maxsecnumdepth{section} % number subsections

\setsecheadstyle{\Large\sffamily\raggedright} % set sans serif section font
\setsubsecheadstyle{\large\sffamily\raggedright} % set sans serif subsection font

%%% Abstract
\setlength{\absleftindent}{0mm}
\setlength{\absrightindent}{0mm}

\renewcommand{\absnamepos}{center}
\setlength{\abstitleskip}{+0cm}




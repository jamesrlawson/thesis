%
%%%%%%%%%%%%%%%%%%%%%%%%%%%%%%%%%%%%%%%%%%%%%%%%%%%%%%%%%%%%%%%%%%%%%%
% Tina Dissertation
% December 2013, modified to Template June 2015
%%%%%%%%%%%%%%%%%%%%%%%%%%%%%%%%%%%%%%%%%%%%%%%%%%%%%%%%%%%%%%%%%%%%%%
% Documentclass Memoir 
% check memman.pdf for help and information
%%%%%%%%%%%%%%%%%%%%%%%%%%%%%%%%%%%%%%%%%%%%%%%%%%%%%%%%%%%%%%%%%%%%%%
\documentclass[12pt,a4paper]{memoir} 
\usepackage{graphicx}
%\usepackage[utf8]{inputenc} % set input encoding to utf8
\usepackage{array} % for tables 
\usepackage{multirow} % for tables 
\usepackage{multicol} % for tables
\usepackage{tabularx} % for tables
\usepackage{booktabs}
\usepackage{cite}
\usepackage{tabularx}
\usepackage[round]{natbib}
\usepackage{threeparttable}
\DisemulatePackage{setspace}
\usepackage{setspace}
\usepackage{longtable}
\usepackage{tabu}
\usepackage{pdflscape}
\usepackage{caption}
 %\useunder{\uline}{\ul}{}

% defines new column type
\newcolumntype{Z}{>{\raggedright\arraybackslash}X}

% add a little vertical padding to cramped tables
\setlength{\extrarowheight}{2pt}


%%%%%%%%%%%%%%%%%%%%%%%%%%%%%%%%%%%%%%%%%%%%%%%%%%%%%%%%%%%%%%%%%%
%%% Examples of Memoir customization
%%% enable, disable or adjust these as desired

%%% PAGE DIMENSIONS
% a4paper is by default 210mm wide and 279 mm wide

% default document in memoir is twoside (recto-verso) and openright (new chapter begins on recto page)

% size of the text area
\settrims{0pt}{0pt}
\settypeblocksize{230mm}{147mm}{*}
\setlength{\spinemargin}{27mm}
\setlength{\foremargin}{36mm}
%\setulmargins{35mm}{45mm}{*}
%\setlength{\marginparwidth}{0mm}
%\setlength{\marginparsep}{0mm}
%\setlength{\textwidth}{140mm}
%\settrimmedsize{0.9\stockheight}{0.9\stockwidth}{*}
%\setlength{\trimtop}{0pt}
%\setlength{\trimedge}{0pt}
%\addtolength{\trimedge}{-\paperwidth}
%\settypeblocksize{*}{\lxvchars}{1.618} % we want to the text block to have golden proportionals
\setulmargins{*}{*}{1.618} % 50pt upper margins
%\setlrmargins{*}{*}{1.3}
\setlrmargins{*}{*}{1} % golden ratio again for left/right margins
\setheaderspaces{*}{*}{1.618}
\checkandfixthelayout % to make sure that the layout parameters make sense

%\addtolength{\textwidth}{0cm}
%\addtolength{\textheight}{1.5cm}
%\addtolength{\textwidth}{-2cm}
%\addtolength{\textheight}{+0.5cm}

%%% \maketitle CUSTOMISATION
% For more than trivial changes, you may as well do it yourself in a titlepage environment
%\pretitle{\begin{center}\sffamily\Huge\MakeUppercase}
%\posttitle{\par\end{center}\vskip 0.5em}

%%% ToC (table of contents) APPEARANCE
\maxtocdepth{subsection} % include subsections
%\renewcommand{\cftchapterpagefont}{}
%\renewcommand{\cftchapterfont}{}     % no bold!

%%% HEADERS & FOOTERS
\pagestyle{Ruled} % try also: empty , plain , headings , ruled , Ruled , companion

%%% CHAPTERS
\chapterstyle{southall} % try also: default , section , hangnum , companion , article, demo

\renewcommand{\chaptitlefont}{\LARGE\sffamily\raggedright} % set sans serif chapter title font
\renewcommand{\chapnumfont}{\LARGE\sffamily\raggedright} % set sans serif chapter number font

%%% TABLES
\newcolumntype{C}[1]{>{\centering}m{#1}} % defines the default layout of the tables (C=centerling, L=left)
\newcolumntype{L}[1]{>{\centering}m{#1}}

%%% SECTIONS
%\hangsecnum % hang the section numbers into the margin to match \chapterstyle{hangnum}
\maxsecnumdepth{section} % number subsections

\setsecheadstyle{\Large\sffamily\raggedright} % set sans serif section font
\setsubsecheadstyle{\large\sffamily\raggedright} % set sans serif subsection font

%%% Abstract
\setlength{\absleftindent}{0mm}
\setlength{\absrightindent}{0mm}

\renewcommand{\absnamepos}{center}
\setlength{\abstitleskip}{+0cm}

%% END Memoir customization
%%%%%%%%%%%%%%%%%%%%%%%%%%%%%%%%%%%%%%%%%%%%%%%%%%%%%%%%%%%%%%%%%%%%%%%%%%%%%%%%%%%%%%%%%%%%%%%%%%%%%%%%%%%%%%%%%%%%%%%%%%%%%%%%%%%%%%%%%%%%%
%%% BEGIN DOCUMENT

\begin{document}

\chapter[Appendix 3 (supplementary to Chapters 2 and 3)]{Appendix 3 (supplementary to Chapters 2 and 3)}

\section*{Study regions and biophysical characteristics of study sites in Chapters 2-3}

This section describes the biophysical characteristics of study sites used in Chapters 2 and 3. For further information about study sites used in Chapter 4, the reader is referred to Arthington et al. (2012).

%%%% FIGURE 1
\begin{figure}[ht]
\begin{center}
\includegraphics[width=12cm,keepaspectratio=true]{Ch6map.png} % figures can be in pdf, png, jpeg or eps format
\caption[Map of study areas described in Chapters 2-4.]{\small{Map showing study areas, and geographical distribution of field sites described in Chapters 2 \& 3 (lower left) and 4 (lower right) (Google Maps 2015).}\label{fig:Ch6_F1}}
%\label{Ch6_F1} % label for cross-referencing
\end{center}
\end{figure}   
\clearpage

\begin{landscape}
\begin{table}[ht]
\tiny
\centering
\caption[Biogeographical attributes of study sites.]{\small{Biogeographical attributes of study sites.}}
\label{biophysical_F1}
{\tabulinesep=1.2mm
%\begin{tabu} to \linewidth {m{3.2cm}m{5cm}X}
\begin{tabu} to \linewidth {m{4cm}m{3cm}m{3cm}m{2cm}m{2cm}m{2cm}m{2cm}}
\hline
\textit{Site}  & \textit{IBRA region}  & \textit{Koppen climate zone} & \textit{Mean annual rainfall (mm)} & \textit{Mean annual temperature (\textsuperscript{o}C)} & \textit{Upstream catchment area (m2)} & \textit{elevation (m asl)} \\
\hline
Mammy Johnsons River at Pikes Crossing & NSW North Coast          & Warm summer, cold winter      & 1136                      & 17.5                         & 158                          & 104               \\
Wallagaraugh River at Princes Highway  & South East Corner        & Warm summer, cold winter      & 925                       & 14.9                         & 477                          & 35                \\
Genoa River at Bondi                   & South East Corner        & Warm summer, cold winter      & 815                       & 13.0                         & 234                          & 417               \\
Wadbilliga River at Wadbilliga         & South East Corner        & Warm summer, cold winter      & 842                       & 14.8                         & 126                          & 201               \\
Tuross River D/S Wadbilliga Junction   & South East Corner        & Warm summer, cold winter      & 843                       & 15.4                         & 918                          & 79                \\
Tuross River at Belowra                & South East Corner        & Warm summer, cold winter      & 831                       & 15.4                         & 564                          & 105               \\
Jacobs River at Jacobs Ladder          & South Eastern Highlands  & Mild-warm summer, cold winter & 563                       & 13.9                         & 184                          & 343               \\
Nariel Creek at Upper Nariel           & South Eastern Highlands  & Mild-warm summer, cold winter & 982                       & 12.8                         & 261                          & 711               \\
Gibbo River at Gibbo Park              & South Eastern Highlands  & Mild-warm summer, cold winter & 919                       & 12.5                         & 390                          & 515               \\
Snowy Creek at Below Granite Flat      & South Eastern Highlands  & Mild-warm summer, cold winter & 1030                      & 13.8                         & 416                          & 331               \\
Mann River at Mitchell                 & New England Tablelands   & Warm summer, cold winter      & 865                       & 16.7                         & 890                          & 401               \\
Cataract Creek at Sandy Hill           & New England Tablelands   & Warm summer, cold winter      & 1019                      & 16.4                         & 237                          & 595               \\
Sportsmans Creek at Gurranang Siding   & South Eastern Queensland & Warm humid summer             & 1094                      & 19.1                         & 205                          & 13                \\
Goodradigbee River at Brindabella      & South Eastern Highlands  & Mild-warm summer, cold winter & 976                       & 12.7                         & 432                          & 510               \\
Jilliby Creek at U/S Wyong River       & Sydney Basin             & Warm summer, cold winter      & 1110                      & 17.7                         & 93                           & 39          \\     
\hline
\end{tabu}}
\end{table}
\end{landscape}
\clearpage





\begin{landscape}
\tiny
{\tabulinesep=1.2mm
%\begin{tabu} to \linewidth {m{3.2cm}m{5cm}X}
\begin{longtabu} to \linewidth {m{3cm}m{1.5cm}XXX}
\caption[Vegetation charactersitics of study sites.]{\small{Vegetation charactersitics of study sites.}}
\label{biophysical_F2}
\hline
\textit{Site}  & \textit{Canopy height (m)} & \textit{Vegetation structure}  & \textit{Dominant species} & \textit{Site history}  \\
\endfirsthead

\hline
\textit{Site}  & \textit{Canopy height (m)} & \textit{Vegetation structure}  & \textit{Dominant species} & \textit{Site history}  \\
\endhead
\hline
Mammy Johnsons River at Pikes Crossing & 15                & Closed canopy, abundant subcanopy and limited groundcover               & \textit{Acmena smithii, Ceratopetalum apetalum, Tristaniopsis laurina}                                                                             & Adjacent smallhold grazing, unfenced, possible historic clearing                                                                                                                                       \\
Wallagaraugh River at Princes Highway  & 12                & Closed low canopy, fern dominated groundcover layer                     & \textit{Tristaniopsis laurina, Doodia aspera}                                                                                                      & Undisturbed, evidenced by numerous large Eucalyptus viminalis individuals adjacent to site                                                                                                             \\
Genoa River at Bondi                   & 30                & Tall open canopy, abundant subcanopy and groundcover                    & \textit{Eucalyptus viminalis, Pomaderris aspera, Leptospermum brevipes, Calochlaena dubia}                                                         & Undisturbed, within South East Forest National Park                                                                                                                                                    \\
Wadbilliga River at Wadbilliga         & 30                & Tall open canopy, abundant shrubs and groundcover                       & \textit{Casuarina cunninghamiana, Eucalyptus cypellocarpa, Acacia floribunda, Melicytus dentatus, Microlaena stipoides}                            & Within Wadbilliga National Park. Mature forest, but historical clearing / pastoral land use possible.                                                                                                  \\
Tuross River D/S Wadbilliga Junction   & 30                & Open forest, shrub and groundcover layers dominant, scattered emergents & \textit{Casuarina cunninghamiana, Eucalyptus elata, Tristaniopsis laurina, Acacia floribunda, Backhousia myrtifolia, Microlaena stipoides}         & Within Wadbilliga National Park. Mature forest with large emergents, but historical clearing / pastoral land usepossible.                                                                              \\
Tuross River at Belowra                & 12                & Riparian scrub with emergent streamside \textit{Casuarina} forest                & \textit{Casuarina cunninghamiana, Acacia floribunda, Leptospermum brevipes}                                                                        & Within Wadbilliga National Park. Mature riparian scrub vegetation, but historical clearing / pastoral land use possible.                                                                               \\
Jacobs River at Jacobs Ladder          & 14                & Open woodland with some emergent Eucalypts                              & \textit{Eucalyptus rubida, Acacia dealbata, Rubus fruticosus}                                                                                      & Within Kosciuzko National Park. In recovery following 2003 fires.                                                                                                                                      \\
Nariel Creek at Upper Nariel           & 40                & Open forest, dense fern groundcover                                     & \textit{Eucalyptus camphora subsp. humeana, Acacia melanoxylon, Rubus fruticosus, Blechnum nudum}                                                  & Mature forest, some clearing adjacent to riparian corridor. Historical clearing / pastoral land use possible.                                                                                          \\
Gibbo River at Gibbo Park              & 35                & Open forest, abundant shrub layer                                       & \textit{Eucalyptus radiata, Acacia dealbata, Lomatia myricoides, Poa labillardierei var labillardierei}                                            & Largely undisturbed, some evidence of fire. Within Alpine National Park. Very large emergent \textit{Eucalyptus viminalis} (\textgreater2m DBH) present.                                                                                       \\ 
Snowy Creek at Below Granite Flat      & 20                & Riparian scrub with low emergent Eucalypts                              & \textit{Eucalyptus camphora subsp. camphora, Eucalyptus stellulata, Coprosma quadrifida, Bursaria spinosa, Leptospermum brevipes}                  & Within Victoria State Forest, clearing for forestry was evident downstream.                                                                                                                            \\
Mann River at Mitchell                 & 30                & Open grassy forest                                                      & \textit{Casuarina cunninghamiana, Eucalyptus ampifolia, Lomandra longifolia, Microlaena stipoides}                                                 & Within Mann River Nature Reserve. Large flood occurred in 2011, historical clearing / pastoral land use is possible.                                                                                   \\
Cataract Creek at Sandy Hill           & 30                & Open forest, abundant shrubs and grassy areas                           & \textit{Casuarina cunninghamiana, Eucalyptus tereticornis, Melicytus dentatus, Lomandra hystrix, Pennisetum clandestinum}                          & Within NSW conservation land, although opposite side of river was cleared pastoral land. Large flood occurred in 2011.                                                                                 \\
Sportsmans Creek at Gurranang Siding   & 25                & Closed forest                                                           & \textit{Lophostemon suaveolens, Alphitonia excelsa, Casuarina glauca, Calochlaena dubia, Oplismenus imbecillis}                                    & Riparian corridor remnant vegetation. Adjacent to Gunnarang State Conservation Area. Large (\textgreater2m DBH) emergent \textit{Lophostemon suaveolens} individuals indicate relatively undisturbed condition. \\
Goodradigbee River at Brindabella      & 25                & Closed forest, abundant shrub layer with sparse groundcover             & \textit{Eucalyptus radiata, Eucalyptus viminalis, Acacia dealbata, Acacia pravissima, Hovea asperifolia subsp. asperifolia, Leptospermum brevipes} & Within Kosciuzko National Park. Fire occurred in 2010.                                                                                                                                                 \\
Jilliby Creek at U/S Wyong River       & 50                & Vine thickets with emergent Eucalypts                                   & \textit{Eucalyptus tereticornis, Eucalyptus resinifera, Commersonia fraseri, Ripogonum album, Lomandra longifolia}                                 & Riparian corridor remnant vegetation. Large emergent \textit{Eucalyptus} individuals (40-50 m tall) indicate relatively undisturbed condition.                                                                 
\hline
\end{longtabu}}
\end{landscape}
\clearpage



PHOTOS

\end{document}









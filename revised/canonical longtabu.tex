%
%%%%%%%%%%%%%%%%%%%%%%%%%%%%%%%%%%%%%%%%%%%%%%%%%%%%%%%%%%%%%%%%%%%%%%
% Tina Dissertation
% December 2013, modified to Template June 2015
%%%%%%%%%%%%%%%%%%%%%%%%%%%%%%%%%%%%%%%%%%%%%%%%%%%%%%%%%%%%%%%%%%%%%%
% Documentclass Memoir 
% check memman.pdf for help and information
%%%%%%%%%%%%%%%%%%%%%%%%%%%%%%%%%%%%%%%%%%%%%%%%%%%%%%%%%%%%%%%%%%%%%%
\documentclass[12pt,a4paper]{memoir} 
\usepackage{graphicx}
%\usepackage[utf8]{inputenc} % set input encoding to utf8
\usepackage{array} % for tables 
\usepackage{multirow} % for tables 
\usepackage{multicol} % for tables
\usepackage{tabularx} % for tables
\usepackage{booktabs}
\usepackage{cite}
\usepackage{tabularx}
\usepackage[round]{natbib}
\usepackage{threeparttable}
\DisemulatePackage{setspace}
\usepackage{setspace}
 \usepackage[normalem]{ulem}
\usepackage{longtable}
\usepackage{tabu}
\usepackage{pdflscape}
 \useunder{\uline}{\ul}{}


% defines new column type
\newcolumntype{Z}{>{\raggedright\arraybackslash}X}

% add a little vertical padding to cramped tables
%\setlength{\extrarowheight}{2pt}
%\setlength{\belowrulesep}{2pt}

%\renewcommand{\arraystretch}{1.5}% array stretch factor

%%%%%%%%%%%%%%%%%%%%%%%%%%%%%%%%%%%%%%%%%%%%%%%%%%%%%%%%%%%%%%%%%%
%%% Examples of Memoir customization
%%% enable, disable or adjust these as desired

%%% PAGE DIMENSIONS
% a4paper is by default 210mm wide and 279 mm wide

% default document in memoir is twoside (recto-verso) and openright (new chapter begins on recto page)

% size of the text area
\settrims{0pt}{0pt}
\settypeblocksize{230mm}{147mm}{*}
\setlength{\spinemargin}{27mm}
\setlength{\foremargin}{36mm}
%\setulmargins{35mm}{45mm}{*}
%\setlength{\marginparwidth}{0mm}
%\setlength{\marginparsep}{0mm}
%\setlength{\textwidth}{140mm}
%\settrimmedsize{0.9\stockheight}{0.9\stockwidth}{*}
%\setlength{\trimtop}{0pt}
%\setlength{\trimedge}{0pt}
%\addtolength{\trimedge}{-\paperwidth}
%\settypeblocksize{*}{\lxvchars}{1.618} % we want to the text block to have golden proportionals
\setulmargins{*}{*}{1.618} % 50pt upper margins
%\setlrmargins{*}{*}{1.3}
\setlrmargins{*}{*}{1} % golden ratio again for left/right margins
\setheaderspaces{*}{*}{1.618}
\checkandfixthelayout % to make sure that the layout parameters make sense

%\addtolength{\textwidth}{0cm}
%\addtolength{\textheight}{1.5cm}
%\addtolength{\textwidth}{-2cm}
%\addtolength{\textheight}{+0.5cm}

%%% \maketitle CUSTOMISATION
% For more than trivial changes, you may as well do it yourself in a titlepage environment
%\pretitle{\begin{center}\sffamily\Huge\MakeUppercase}
%\posttitle{\par\end{center}\vskip 0.5em}

%%% ToC (table of contents) APPEARANCE
\maxtocdepth{subsection} % include subsections
%\renewcommand{\cftchapterpagefont}{}
%\renewcommand{\cftchapterfont}{}     % no bold!

%%% HEADERS & FOOTERS
\pagestyle{Ruled} % try also: empty , plain , headings , ruled , Ruled , companion

%%% CHAPTERS
\chapterstyle{southall} % try also: default , section , hangnum , companion , article, demo

\renewcommand{\chaptitlefont}{\LARGE\sffamily\raggedright} % set sans serif chapter title font
\renewcommand{\chapnumfont}{\LARGE\sffamily\raggedright} % set sans serif chapter number font

%%% TABLES
\newcolumntype{C}[1]{>{\centering}m{#1}} % defines the default layout of the tables (C=centerling, L=left)
\newcolumntype{L}[1]{>{\centering}m{#1}}

%%% SECTIONS
%\hangsecnum % hang the section numbers into the margin to match \chapterstyle{hangnum}
\maxsecnumdepth{section} % number subsections

\setsecheadstyle{\Large\sffamily\raggedright} % set sans serif section font
\setsubsecheadstyle{\large\sffamily\raggedright} % set sans serif subsection font

%%% Abstract
\setlength{\absleftindent}{0mm}
\setlength{\absrightindent}{0mm}

\renewcommand{\absnamepos}{center}
\setlength{\abstitleskip}{+0cm}

%% END Memoir customization

%%%%%%%%%%%%%%%%%%%%%%%%%%%%%%%%%%%%%%%%%%%%%%%%%%%%%%%%%%%%%%%%%%%%%%%%%%%%%%%%%%%%%%%%%%%%%%%%%%%%%%%%%%%%%%%%%%%%%%%%%%%%%%%%%%%%%%%%%%%%%
%%% BEGIN DOCUMENT

\begin{document}
\doublespacing

% Please add the following required packages to your document preamble:
% \usepackage[normalem]{ulem}
% \useunder{\uline}{\ul}{}

% Please add the following required packages to your document preamble:
% \usepackage[normalem]{ulem}
% \useunder{\uline}{\ul}{}

\pagestyle{empty}
\begin{landscape}
\begin{tiny}
{\tabulinesep=1.2mm
\begin{longtabu} to \linewidth {p{3cm}XXX} 
\caption{blah} \\
\toprule
% -----------------These are headings----------------------------------%
\textit{Trait} & \textit{Definition} & \textit{Functional responses} \& \textit{inherent trade-offs} & \textit{Functional effects} \\ \hline
%
\endfirsthead
%
%\multicolumn{4}{c}%
%{{\bfseries  Continued from previous page}} \\
%\hline
%
\toprule
\textit{Trait} & \textit{Definition} & \textit{Functional responses} \& \textit{inherent trade-offs} & \textit{Functional effects} \\ \hline
\endhead
%
%\hline \multicolumn{4}{|r|}{{Continued on next page}} \\ \hline
%\endfoot
%
%\hline
%\multicolumn{4}{|r|}{{Concluded}} \\ \hline
%\endlastfoot
%-----------Headings end---------------------------------
\hline
Growth form & Categorical description of morphology: tree, shrub, woody climber, herbaceous climber, graminoid, herb. & Differential responses to mechanical and biochemical stresses associated caused by flooding; different strategies for coping with drought and heat stress. & Differential biogeomorphic effects on fluvial landform cohesion and sediment deposition. \\ \hline
Specific leaf area (SLA) & Ratio of one-sided leaf area to oven dry mass (cm2 / g). & SLA is associated with leaf construction cost, photosynthetic rate and carbon : nitrogen economics. Indicator of  ecological strategy under favourable vs. stressful conditions(Wright et al. 2004). & Affects ecosystem productivity and nutrient recycling (Wright et al. 2004). \\ \hline
Leaf area & One-sided leaf area (cm2). & Shade tolerance (larger leaves) vs. enhanced thermal regulation ability in hot, dry conditions (smaller leaves) (Cornelissen et al. 2003). & May influence flow resistance of vegetation (and therefore fluvial erosion / deposition) when inundated. \\ \hline
Maximum canopy height & Height above ground of apical meristem (m). & Affects ability to tolerate mechanical disturbances such as flooding and maintain xylem integrity in dry conditions (Westoby \& Wright 2006). & Determines coarse physical structure of plant community. Surrogate for competitive ability: taller plants receive more light but must construct and maintain support structures (Falster 2006). \\ \hline
Seed mass & Combined mass of the seed coat, endosperm and embryo (g). Excludes dispersal structures. & Larger seed mass confers ability to establish in unfavourable conditions (Leishman et al. 2000). Also related to seed buoyancy (Carthey 2014, unpublished data). & Seeds may be an important food source for animals. \\ \hline
Wood density & Oven dry mass divided by green volume (g/cm3) & Dense wood tissue confers mechanical strength, but is energetically expensive to construct. Wood density influences ability to tolerate drought stress and disturbance (Telewski 1995; Preston, Cornwell \& Denoyer 2006; Lawson et al. 2015). & Regulates decomposition rate; this affects nutrient cycling and determines the residency time of woody debris in the fluvial system (Mackensen, Bauhus \& Webber 2003). \\ \hline
Flowering period length & Proportion of the year spent in flower (proportion, dimensionless). & Indicates species’ ability to respond reproductively to favourable conditions. & Flowers may be an important food source for animals. \\ \hline

\label{my-label}
\end{longtabu}}
\end{landscape}
\end{tiny}
\pagestyle{myheadings}

\end{document}




Species richness was highest when minimum flow conditions were unevenly distributed throughout the year (C\_MinM, R\textsuperscript{2}  $=$ 0.237, Fig. \ref{fig:Ch4_F2}c), and where these seasonal patterns of minimum flows were consistent between years (M\_MinM, R\textsuperscript{2}  $=$ 0.129, Fig. \ref{fig:Ch4_F2}d). Richness declined with increasing duration of high flow periods (HSMeanDur, R\textsuperscript{2}  $=$ 0.290, Fig. \ref{fig:Ch4_F2}e), but increased somewhat as these high flow periods became more frequent (MDFAnnHSNum, R\textsuperscript{2}  $=$ 0.106, Fig. \ref{fig:Ch4_F2}f). Increased dry season flows due to flow modification were weakly associated reduced species richness (MDFMDFDry.mod, R\textsuperscript{2}  $=$ 0.117, Fig. \ref{fig:Ch4_F2}g). Alterations to seasonal consistency of minimum flow patterns had a strong effect (M\_MinM.mod, R\textsuperscript{2}  $=$ 0.412, Fig. \ref{fig:Ch4_F2}h), and corroborated the trend observed in Fig. \ref{fig:Ch4_F2}d: species richness increased as patterns of monthly minimum flows became more consistent throughout the hydrological record. With respect to climate, species richness was greater at sites which experienced higher rainfall (clim\_pwet, R\textsuperscript{2}  $=$ 0.390, Fig. \ref{fig:Ch4_F2}i) and less variable temperature regimes (clim\_tsea, R\textsuperscript{2}  $=$ 0.349, Fig. \ref{fig:Ch4_F2}j). Soils which contained more organic carbon (soil\_soc, R\textsuperscript{2}  $=$ 0.202, Fig. \ref{fig:Ch4_F2}j) and higher silt content (soil\_slt, R\textsuperscript{2}  $=$ 0.239, Fig. \ref{fig:Ch4_F2}k), lower total phosphorus (soil\_pto, R\textsuperscript{2}  $=$ 0.110, Fig. \ref{fig:Ch4_F2}l) and lower available water capacity (soil\_awc, R\textsuperscript{2}  $=$ 0.203, Fig. \ref{fig:Ch4_F2}m) supported richer communities.

 
Variation in species richness was well explained by modification of seasonal consistency of minimum flow patterns (M\_MinM.mod, R\textsuperscript{2}  $=$ 0.379, Fig. \ref{fig:Ch4_F2}c), although species richness was only weakly related to M_MinM itself (R\textsuperscript{2} $=$ 0.134). Species richness showed a humped relationship with the proportion of upstream catchment used in dryland agricultural production (production\_dryland, R\textsuperscript{2}  $=$ 0.224, Fig. \ref{fig:Ch4_F2}d), and a negative relationship with irrigated agricultural production (production\_irrigated, R\textsuperscript{2}  $=$ 0.135, Fig. \ref{fig:Ch4_F2}e). With respect to climate, species richness was greater at sites which experienced higher rainfall in both dry (clim\_pdry, R\textsuperscript{2}  $=$ 0.417, Fig. \ref{fig:Ch4_F2}e) and wet seasons (clim\_pwet, R\textsuperscript{2}  $=$ 0.465, Fig. \ref{fig:Ch4_F2}f). Lower soil cation exchange capacity (soil\_ece, R\textsuperscript{2}  $=$ 0.402, Fig. \ref{fig:Ch4_F2}g) and bulk density (soil\_bdw, R\textsuperscript{2}  $=$ 0.208, Fig. \ref{fig:Ch4_F2}h) supported richer communities.


The data did not support hypothesis 1, that rivers with more heterogeneous flow regimes support communities with higher species richness, or hypothesis 3, that there is a unimodal relationship between species richness and flow heterogeneity. 

Support for hypothesis 4 (that species richness and functional diversity should decrease along gradients of increasing flow modification and catchment land-use intensity) was mixed. Greater species richness associated with anthropogenic homogenisation of seasonal consistency of minimum flows (M_MinM) in fact lends evidence to the counter-hypothesis. Although several other significant relationships were found between species richness and metrics of flow modification 

Further, these results contradict hypothesis 4 (that species richness and functional diversity should decrease along gradients of increasing flow modification and catchment land-use intensity), given that rivers which experienced more consistent patterns of minimum flows hosted richer plant communities.